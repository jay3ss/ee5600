\begin{question}{5}
Find the unit-step response of the following system using two different methods.
\begin{align}
\dot{X}(t) =
  \begin{bmatrix}
       0 &  1 \\
      -2 & -2
  \end{bmatrix}
    X(t) +
  \begin{bmatrix}
    1 \\ 1
  \end{bmatrix}
  u(t) \nonumber
\end{align}

\begin{equation} \label{eq:ssr_output}
y(t) = \begin{bmatrix} 2 & 3\end{bmatrix}X(t)
\end{equation}

[Note: I'm going to be assuming that initial conditions are all zero.]
\end{question}

\noindent \textbf{a)} Using the Laplace Transform.

\begin{equation} \label{eq:laplace_ssr}
Y(s) = [C (s\mathbb{I} - A)^{-1} B + D] U(s)
\end{equation}

Equation \ref{eq:laplace_ssr} gives the Laplace Transform of the output
equation, Equation \ref{eq:ssr_output}, where

\begin{equation}
 A =
 \begin{bmatrix}
      0 &  1 \\
     -2 & -2
 \end{bmatrix}, \quad
 B =
 \begin{bmatrix}
   1 \\ 1
 \end{bmatrix}, \quad
 C = \begin{bmatrix} 2 & 3\end{bmatrix}, \quad
 D = 0 \nonumber
\end{equation}

If we simplify Equation \ref{eq:laplace_ssr}, we arrive at

\begin{equation} \label{eq:laplace_output_tf}
Y(s) = T(s) U(s) \nonumber
\end{equation}

\noindent where

\begin{align}
  T(s) &= C (s\mathbb{I} - A)^{-1} B + D \nonumber \\
       &=
       \begin{bmatrix}
        2 & 3
       \end{bmatrix}
       \begin{bmatrix}
        s & -1 \\
        2 & s + 2
      \end{bmatrix}^{-1}
      \begin{bmatrix}
       1 \\ 1
      \end{bmatrix} \nonumber \\
      &=
      \begin{bmatrix}
       2 & 3
      \end{bmatrix}
      \Bigg(\frac{1}{s^2 + 2s + 2} \Bigg)
      \begin{bmatrix}
       s + 2 & 1 \\
       -2     & s
      \end{bmatrix}
      \begin{bmatrix}
       1 \\ 1
      \end{bmatrix} \nonumber \\
      &= \frac{5s}{s^2 + 2s + 2} \label{eq:laplace_tf}
\end{align}

With $U(s) = \dfrac{1}{s}$ and Equations \label{eq:laplace_output_tf} and
\label{eq:laplace_tf} we have

\begin{align}
  Y(s) &= \frac{5s}{s^2 + 2s + 2} \cdot \frac{1}{s} \nonumber \\
       &= \frac{5}{s^2 + 2s + 2} \nonumber \\
       &= \frac{5}{(s + 1)^2 + 1} \label{eq:laplace_ys}
\end{align}

\noindent and by taking the inverse Laplace Transform of Equation
\ref{eq:laplace_ys}, we finally arrive at

\begin{equation} \label{eq:5_solved_laplace}
  y(t) = 5e^{-t} \sin (t)
\end{equation}

\noindent \textbf{b)} Using the Cayley-Hamilton theorem.

The general solution for a linear time-invariant (LTI) system is

\begin{equation} \label{eq:lti_gen_sol}
  X(t) = e^{At}X(0) + \int_0^{\infty} e^{A(t - \tau)} B U(\tau) d \tau
\end{equation}

\noindent Using the Cayley-Hamilton theorem, $e^{At}$ can be found. First, the
eigenvalues must be determined.

\begin{align}
  \begin{vmatrix}
    \lambda \mathbb{I} - A
  \end{vmatrix} \nonumber
  &=
  \begin{vmatrix}
    \lambda & -1 \\
    2       & \lambda + 2
  \end{vmatrix} \\
  &= \lambda (\lambda + 2) - 2 (-1) \nonumber \\
  &= \lambda^2 + 2 \lambda + 2 = 0 \nonumber \\
  \therefore \lambda_{1,2} &= -1 \pm j \label{eq:q5_lambda}
\end{align}

% \label{eq:cayley_hamilton_sol_x}
\begin{equation}
  \text{let} \quad
  f(\lambda) = e^{\lambda t} =
  h(\lambda) = \beta_0 + \beta_1 \lambda
\end{equation}

\noindent Plugging in the values for $\lambda_1$ and $\lambda_2$ we arrive at

\begin{align}
  e^{t(-1 + j)} &= \beta_0 + \beta_1 (-1 + j) \nonumber \\
  e^{t(-1 - j)} &= \beta_0 + \beta_1 (-1 - j)  \nonumber
\end{align}

Solving the above system yields

\begin{align}
  \beta_0 &= e^{-t} \sin t \label{eq:q5_beta0} \\
  \beta_1 &= e^{-t}(\cos t + \sin t) \label{eq:q5_beta1}
\end{align}

\noindent By the Cayley-Hamilton theorem, $h(\lambda) \mapsto h(A)$ and
$f(\lambda) \mapsto f(A)$ which yields

\begin{align}
h(A) &= \beta_0 \mathbb{I} + \beta_1 A \nonumber \\
     &=
   \begin{bmatrix}
     e^{-t} [\cos t + \sin t] & e^{-t} \sin t \\
     -2 e^{-t} \sin t & e^{-t} [\cos t - \sin t]
   \end{bmatrix} \nonumber \\
f(A) &= e^{At} \nonumber \\
f(A) &= h(A) \nonumber \\
e^{At} &=
   \begin{bmatrix}
     e^{-t} [\cos t + \sin t] & e^{-t} \sin t \\
     -2 e^{-t} \sin t & e^{-t} [\cos t - \sin t]
   \end{bmatrix}
\end{align}

By solving Equation \ref{eq:y_state_space_gen} the solution to the system will be
found.

\begin{equation} \label{eq:y_state_space_gen}
  y(t) = \int_{0}^{t} C e^{A(t-\tau)} B U(\tau) d \tau
\end{equation}

\noindent Simplifying the integrand of Equation \ref{eq:y_state_space_gen} gives

\begin{align} \label{eq:5_solved_cayley_hamilton}
  y(t) &= \int_{0}^{t} 5e^{-(t-\tau)} (\cos(t-\tau) - \sin(t-\tau)) d \tau \nonumber \\
  y(t) &= 5e^{-t} \sin(t)
\end{align}

\noindent By comparing Equations \ref{eq:5_solved_laplace} and
\ref{eq:5_solved_cayley_hamilton}, we can see that they are the same.


% To make calculations a bit easier,

% The equation for $e^{-A(t - \tau)}$ is found by substituting $t = t - \tau$ as
% shown in Equation \ref{eq:e_attau}.
%
% \begin{equation}\label{eq:e_attau}
%   e^{A(t - \tau)} =
%        \begin{bmatrix}
%          e^{-(t - \tau)} [\cos (t - \tau) + \sin (t - \tau)] & e^{-(t - \tau)} \sin (t - \tau) \\
%          -2 e^{-(t - \tau)} \sin (t - \tau) & e^{-(t - \tau)} [\cos (t - \tau) - \sin (t - \tau)]
%        \end{bmatrix}
% \end{equation}
