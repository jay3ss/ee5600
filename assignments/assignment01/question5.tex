\begin{question}{5}
Find the unit-step response of the following system using two different methods.
\begin{align}
\dot{X}(t) =
  \begin{bmatrix}
       0 &  1 \\
      -2 & -2
  \end{bmatrix}
    X(t) +
  \begin{bmatrix}
    1 \\ 1
  \end{bmatrix}
  u(t) \nonumber
\end{align}

\begin{equation} \label{eq:ssr_output}
y(t) = \begin{bmatrix} 2 & 3\end{bmatrix}X(t)
\end{equation}

[Note: I'm going to be assuming that initial conditions are all zero.]
\end{question}

\noindent \textbf{a)} Using the Laplace Transform.

\begin{equation} \label{eq:laplace_ssr}
Y(s) = [C (s\mathbb{I} - A)^{-1} B + D] U(s)
\end{equation}

Equation \ref{eq:laplace_ssr} gives the Laplace Transform of the output
equation, Equation \ref{eq:ssr_output}, where

\begin{equation}
 A =
 \begin{bmatrix}
      0 &  1 \\
     -2 & -2
 \end{bmatrix}, \quad
 B =
 \begin{bmatrix}
   1 \\ 1
 \end{bmatrix}, \quad
 C = \begin{bmatrix} 2 & 3\end{bmatrix}, \quad
 D = 0 \nonumber
\end{equation}

If we simplify Equation \ref{eq:laplace_ssr}, we arrive at

\begin{equation} \label{eq:laplace_output_tf}
Y(s) = T(s) U(s) \nonumber
\end{equation}

\noindent where

\begin{align}
  T(s) &= C (s\mathbb{I} - A)^{-1} B + D \nonumber \\
       &=
       \begin{bmatrix}
        2 & 3
       \end{bmatrix}
       \begin{bmatrix}
        s & -1 \\
        2 & s + 2
      \end{bmatrix}^{-1}
      \begin{bmatrix}
       1 \\ 1
      \end{bmatrix} \nonumber \\
      &=
      \begin{bmatrix}
       2 & 3
      \end{bmatrix}
      \Bigg(\frac{1}{s^2 + 2s + 2} \Bigg)
      \begin{bmatrix}
       s + 2 & 1 \\
       -2     & s
      \end{bmatrix}
      \begin{bmatrix}
       1 \\ 1
      \end{bmatrix} \nonumber \\
      &= \frac{5s}{s^2 + 2s + 2} \label{eq:laplace_tf}
\end{align}

With $U(s) = \dfrac{1}{s}$ and Equations \label{eq:laplace_output_tf} and
\label{eq:laplace_tf} we have

\begin{align}
  Y(s) &= \frac{5s}{s^2 + 2s + 2} \cdot \frac{1}{s} \nonumber \\
       &= \frac{5}{s^2 + 2s + 2} \nonumber \\
       &= \frac{5}{(s + 1)^2 + 1} \label{eq:laplace_ys}
\end{align}

\noindent and by taking the inverse Laplace Transform of Equation
\ref{eq:laplace_ys}, we finally arrive at

\begin{equation}
  y(t) = 5e^{-t} \sin (t)
\end{equation}

\noindent \textbf{b)} Using the Cayley-Hamilton theorem.

The general solution for a linear time-invariant (LTI) system is

\begin{equation} \label{eq:lti_gen_sol}
  X(t) = e^{At}X(0) + \int_0^{\infty} e^{-A(\tau -t)} B U(\tau) d \tau
\end{equation}

\noindent Using the Cayley-Hamilton theorem, $e^{-At}$ and $e^{-A(\tau - t)}$
can be found.
