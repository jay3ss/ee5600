\begin{question}{1} %You can use theorem, exercise, problem, or question here.  Modify x.yz to be whatever number you are proving
\begin{equation}
x_1 = \begin{bmatrix}2 \\ -3 \\ -1 \end{bmatrix}, \quad x_2 = \begin{bmatrix}1 \\ 1 \\ -1 \end{bmatrix} \nonumber
\end{equation}
\end{question}

\noindent\textbf{a)} \newline
First norm:

\begin{align*}
    ||x_1||_1 &= \sum_{1}^{3} x_{1i} \\
              &= 2 - 3 - 1 \\
              &= \textbf{-2}
\end{align*}

\noindent and

\begin{align*}
    ||x_2||_1 &= \sum_{1}^{3} x_{2i} \\
              &= 1 + 1 - 1 \\
              &= \textbf{1}
\end{align*}


\noindent\textbf{b)} \newline
Second norm:

\begin{align*}
    ||x_1||_2 &= \sqrt{\sum_{1}^{3} x_{1i}^2} \\
              &= \sqrt{2^2 + (- 3)^2 + (- 1)^2} \\
              &= \bm{\sqrt{14}}
\end{align*}

\noindent and


\begin{align*}
    ||x_2||_2 &= \sqrt{\sum_{1}^{3} x_{2i}^2} \\
              &= \sqrt{1^2 + 1^2 + (- 1)^2} \\
              &= \bm{\sqrt{3}}
\end{align*}


\noindent\textbf{c)} \newline
Infinite norm:

\begin{align*}
    ||x_1||_{\infty} &= max(x_1) \\
                     &= \bm{2}
\end{align*}

\noindent and


\begin{align*}
    ||x_2||_{\infty} &= max(x_2) \\
                     &= \bm{1}
\end{align*}
