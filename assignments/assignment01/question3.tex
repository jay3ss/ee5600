\begin{question}{3} %You can use theorem, exercise, problem, or question here.  Modify x.yz to be whatever number you are proving
\begin{equation}
A_1 =
\begin{bmatrix}
     0 & 1 &  0 \\
     0 & 0 &  0 \\
     0 & 0 & -1
\end{bmatrix}, \quad
A_2 =
\begin{bmatrix}
     4 & 1 & 1 \\
     3 & 2 & 0 \\
     1 & 1 & 0
\end{bmatrix}, \quad
A_3 =
\begin{bmatrix}
    1 &  2 & -3 & 4 \\
    0 & -1 &  2 & 2 \\
    0 &  0 &  0 & 1
\end{bmatrix}  \nonumber
\end{equation}
\end{question}

\noindent $\mathbf{A_1}$:

By examination, it can be seen that the matrix $A_1$ has two linearly 
independent columns. Therefore, the \textbf{rank of} 
$\mathbf{A_1}$ \textbf{is 2}. There are four columns in $A_1$ and its rank is 
two, therefore $\mathbf{A_1}$\textbf{'s nullity is} $\mathbf{4 - 2 = 2}$.

$\\$

\noindent $\mathbf{A_2}$:

Matrix $A_2$ can be transformed into an upper triangle using a sequence of
elementary transformations as demonstrated by \cite{chen1998linear} and is given
by Equation \ref{eq:q3_a2_upper_tri}. According to \cite{chen1998linear}, the 
rank of an upper triangular matrix is equal to the number of nonzero rows. 
The matrix $A_{2ref}$ has three nonzero rows and therefore it and $A_2$ have
a \textbf{rank of 3}. The \textbf{nullity of} $\mathbf{A_2}$ \textbf{is 0.}

\begin{equation} \label{eq:q3_a2_upper_tri}
  A_2 \longrightarrow 
  \begin{bmatrix}
  1 &  1 & 0 \\
  0 & -1 & 0 \\
  0 &  0 & 1
  \end{bmatrix}
  = A_{2ref}
\end{equation}

$\\$

\noindent $\mathbf{A_3}$:

By examination, it can be seen that the matrix $A_3$ has three linearly 
independent columns. Therefore, the \textbf{rank of} 
$\mathbf{A_3}$ \textbf{is 3}. There are four columns in $A_3$ and its rank is 
three, therefore $\mathbf{A_3}$\textbf{'s nullity is} $\mathbf{4 - 3 = 1}$.
