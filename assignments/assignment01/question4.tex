\begin{question}{4} %You can use theorem, exercise, problem, or question here.  Modify x.yz to be whatever number you are proving
\begin{equation}
A_1 = 
\begin{bmatrix} 
  1 & 1 & 0 \\ 
  0 & 0 & 1 \\
  0 & 0 & 1
\end{bmatrix} \nonumber
\end{equation}
\end{question}

The Cayley-Hamilton theorem can be used to compute powers of a matrix 
\cite{chen1998linear}. First, the eigenvalues of the matrix $A_1$ must be found.

\begin{align}
\begin{vmatrix}
{\mathbf{I}\lambda - A_1}
\end{vmatrix} &= 
  \begin{vmatrix}
  \lambda -1 & -1 & 0 \\
  0 & \lambda & -1 \\
  0 & 0 & 1 - \lambda
  \end{vmatrix} \nonumber \\
  &= \lambda (\lambda - 1)^2 \nonumber \\
\lambda &= 0, 1, 1
\end{align}

Now that the eigenvalues have been found, different powers of $A_1$ can be found
by finding the $\beta_i$s in Equation \ref{eq:cayley_hamilton_scalar}

\begin{equation}\label{eq:cayley_hamilton_scalar}
  h(\lambda_i) = \beta_0 + \beta_1 \lambda_i + \beta_2 \lambda_i^2
\end{equation}

\noindent using the previously found eigenvalues each corresponding to a 
$\lambda_i$ . There is a slight modification that needs to be made in order to 
deal with the repeated eigenvalues. We can use a derivative of Equation 
\ref{eq:cayley_hamilton_scalar} to allow us to solve for the coefficients

\begin{equation}
\frac{d h(\lambda_i)}{d\lambda_i} = \beta_1  + 2 \beta_2 \lambda_i
\end{equation}

\noindent Once we know our coefficients ($\beta_i$) we can use the following
formula to find any power of $A_i$

\begin{equation}\label{eq:cayley_hamilton_matrix}
  h(A_1) = \beta_0 + \beta_1 A_1 + \beta_2 A_1^2
\end{equation}

\noindent \textbf{a)} Find $A_1^{10}$.

Here, $h(\lambda_i) = \lambda_i^{10}$ and 
$\dfrac{d h(\lambda_i)}{d\lambda_i} = 10 \lambda_i^9$. For $\lambda_i = 0$ 


\begin{align}
  (0)^{10} &= \beta_0 + \beta_1 (0) + \beta_2 (0)^2 \nonumber \\
         0 &= \beta_0 
\end{align}

For $\lambda_i = 1$ using $h(\lambda_i)$


\begin{align}
  (1)^{10} &= \beta_0 + \beta_1 (1) + \beta_2 (1)^2 \nonumber \\
         1 &= 0 + \beta_1 + \beta_2 \nonumber \\
         1 &= \beta_1 + \beta_2
\end{align}

For $\lambda_i = 1$ using $\dfrac{d h(\lambda_i)}{d \lambda_i}$


\begin{align}
  10(1)^{9} &=  \beta_1 + 2 \beta_2 (1) \nonumber \\
         10 &= 0 + \beta_1 + 2 \beta_2 \nonumber \\
         10 &= \beta_1 + 2 \beta_2
\end{align}

And solving the system of equations we get

\begin{align}
  \beta_0 &= 0  \nonumber \\
  \beta_1 &= -8 \\
  \beta_2 &= 9  \nonumber
\end{align}

Now the matrix form of the Cayley-Hamilton theorm, 
Equation \ref{eq:cayley_hamilton_matrix}, can be prepared, then utilized to find
$A_1^{10}$

\begin{align}
  h(A_1)   &= \beta_0 + \beta_1 A_1 + \beta_2 A_1^2 \nonumber \\
  A_1^{10} &= -8 A_1 + 9 A_1^2 \nonumber \\
           &= -8 \begin{bmatrix} 
                  1 & 1 & 0 \\ 
                  0 & 0 & 1 \\
                  0 & 0 & 1
                \end{bmatrix} \nonumber
              + 9 \begin{bmatrix} 
                  1 & 1 & 0 \\ 
                  0 & 0 & 1 \\
                  0 & 0 & 1
                \end{bmatrix}^2 \nonumber \\
           &= -8 \begin{bmatrix} 
                  1 & 1 & 0 \\ 
                  0 & 0 & 1 \\
                  0 & 0 & 1
                \end{bmatrix} \nonumber
              + 9 \begin{bmatrix} 
                  1 & 1 & 1 \\ 
                  0 & 0 & 1 \\
                  0 & 0 & 1
                \end{bmatrix} \nonumber \\
           &= \mathbf{
           \begin{bmatrix} 
              1 & 1 & 9 \\ 
              0 & 0 & 1 \\
              0 & 0 & 1
           \end{bmatrix}
           } \nonumber \\
\end{align}

\noindent \textbf{b)} Find $A_1^{103}$

Here $h(\lambda_i) = \lambda_i^{103}$, 
$\dfrac{d h(\lambda_i)}{d \lambda_i} = 103 \lambda_i^{102}$, and 
$h(A_1) = A_1^{103}$.

For $\lambda_i = 0$

\begin{align}
  (0)^{103} &= \beta_0 + \beta_1 (0) + \beta_2 (0)^2 \nonumber \\
         0  &= \beta_0 
\end{align}

For $\lambda_i = 1$ using $h(\lambda_i)$


\begin{align}
  (1)^{103} &= \beta_0 + \beta_1 (1) + \beta_2 (1)^2 \nonumber \\
          1 &= 0 + \beta_1 + \beta_2 \nonumber \\
          1 &= \beta_1 + \beta_2
\end{align}

For $\lambda_i = 1$ using $\dfrac{d h(\lambda_i)}{d \lambda_i}$


\begin{align}
  103(1)^{102} &=  \beta_1 + 2 \beta_2 (1) \nonumber \\
           103 &= 0 + \beta_1 + 2 \beta_2 \nonumber \\
           103 &= \beta_1 + 2 \beta_2
\end{align}

And solving the system of equations we get

\begin{align}
  \beta_0 &= 0  \nonumber \\
  \beta_1 &= -101 \\
  \beta_2 &= 102  \nonumber
\end{align}

Now the matrix form of the Cayley-Hamilton theorm, 
Equation \ref{eq:cayley_hamilton_matrix}, can be prepared, then utilized to find
$A_1^{103}$

\begin{align}
  h(A_1)   &= \beta_0 + \beta_1 A_1 + \beta_2 A_1^2 \nonumber \\
  A_1^{10} &= -101 A_1 + 102 A_1^2 \nonumber \\
           &= -101 \begin{bmatrix} 
                  1 & 1 & 0 \\ 
                  0 & 0 & 1 \\
                  0 & 0 & 1
                \end{bmatrix} \nonumber
              + 102 \begin{bmatrix} 
                  1 & 1 & 1 \\ 
                  0 & 0 & 1 \\
                  0 & 0 & 1
                \end{bmatrix} \nonumber \\
           &= \mathbf{
           \begin{bmatrix} 
              1 & 1 & 102 \\ 
              0 & 0 & 1 \\
              0 & 0 & 1
           \end{bmatrix}
           } \nonumber \\
\end{align}


\noindent \textbf{c)} Find $e^{A_1 t}$

Here $h(\lambda_i) = e^{\lambda_i t}$, 
$\dfrac{d h(\lambda_i)}{d \lambda_i} = te^{\lambda_i t}$, and 
$h(A_1) = e^{A_1 t}$.

For $\lambda_i = 0$

\begin{align}
  e^0 &= \beta_0 + \beta_1 (0) + \beta_2 (0)^2 \nonumber \\
    1 &= \beta_0 
\end{align}

For $\lambda_i = 1$ using $h(\lambda_i)$


\begin{align}
  e^{(1)t} &= \beta_0 + \beta_1 (1) + \beta_2 (1)^2 \nonumber \\
       e^t &= 1 + \beta_1 + \beta_2
\end{align}

For $\lambda_i = 1$ using $\dfrac{d h(\lambda_i)}{d \lambda_i}$


\begin{align}
  te^{(1)t} &= \beta_1 + 2 \beta_2 (1) \nonumber \\
       te^t &= \beta_1 + 2 \beta_2
\end{align}

And solving the system of equations we get

\begin{align}
  \beta_0 &= 1  \nonumber \\
  \beta_1 &= e^t (2 - t) \\
  \beta_2 &= e^t (t -1) + 1 \nonumber
\end{align}

Now the matrix form of the Cayley-Hamilton theorm, 
Equation \ref{eq:cayley_hamilton_matrix}, can be prepared, then utilized to find
$A_1^{103}$

\begin{align}
  h(A_1)   &= \beta_0 + \beta_1 A_1 + \beta_2 A_1^2 \nonumber \\
  A_1^{10} &= 1 \mathbb{I} + e^t (2 - t) A_1 + [e^t (t -1) + 1] A_1^2 \nonumber \\
           &= \mathbb{I}
              + \begin{bmatrix} 
                  e^t (2 - t) & e^t (2 - t) & 0 \\ 
                  0 & 0 & e^t (2 - t) \\
                  0 & 0 & e^t (2 - t)
                \end{bmatrix} \nonumber
              + \begin{bmatrix} 
                  e^t (t -1) + 1 & e^t (t -1) + 1 & e^t (t -1) + 1 \\ 
                  0 & 0 & e^t (t -1) + 1 \\
                  0 & 0 & e^t (t -1) + 1
                \end{bmatrix} \nonumber \\
           &= \mathbf{
           \begin{bmatrix} 
              e^t + 2 & e^t + 1 & e^t + 1 \\ 
              0 & 1 & e^t + 1 \\
              0 & 0 & e^t + 2
           \end{bmatrix}
           } \nonumber \\
\end{align}



