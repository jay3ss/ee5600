% --------------------------------------------------------------
% This is all preamble stuff that you don't have to worry about.
% Head down to where it says "Start here"
% --------------------------------------------------------------
 
\documentclass[12pt]{article}
 
\usepackage[margin=1in]{geometry} 
\usepackage{amsmath,amsthm,amssymb}
\usepackage{bm}

\newcommand{\N}{\mathbb{N}}
\newcommand{\Z}{\mathbb{Z}}
 
\newenvironment{theorem}[2][Theorem]{\begin{trivlist}
\item[\hskip \labelsep {\bfseries #1}\hskip \labelsep {\bfseries #2.}]}{\end{trivlist}}
\newenvironment{lemma}[2][Lemma]{\begin{trivlist}
\item[\hskip \labelsep {\bfseries #1}\hskip \labelsep {\bfseries #2.}]}{\end{trivlist}}
\newenvironment{exercise}[2][Exercise]{\begin{trivlist}
\item[\hskip \labelsep {\bfseries #1}\hskip \labelsep {\bfseries #2.}]}{\end{trivlist}}
\newenvironment{problem}[2][Problem]{\begin{trivlist}
\item[\hskip \labelsep {\bfseries #1}\hskip \labelsep {\bfseries #2.}]}{\end{trivlist}}
\newenvironment{question}[2][Question]{\begin{trivlist}
\item[\hskip \labelsep {\bfseries #1}\hskip \labelsep {\bfseries #2.}]}{\end{trivlist}}
\newenvironment{corollary}[2][Corollary]{\begin{trivlist}
\item[\hskip \labelsep {\bfseries #1}\hskip \labelsep {\bfseries #2.}]}{\end{trivlist}}
 
\begin{document}
 
% --------------------------------------------------------------
%                         Start here
% --------------------------------------------------------------
 
\title{EE 5600: Linear Systems Analysis - Assignment 1}%replace X with the appropriate number
\author{Joshua Saunders\\ %replace with your name
} %if necessary, replace with your course title
 
\maketitle
 
\begin{question}{1} %You can use theorem, exercise, problem, or question here.  Modify x.yz to be whatever number you are proving
\begin{equation}
x_1 = \begin{bmatrix}2 \\ -3 \\ -1 \end{bmatrix}, \quad x_2 = \begin{bmatrix}1 \\ 1 \\ -1 \end{bmatrix} \nonumber
\end{equation}
\end{question}
 
\noindent\textbf{a)} \newline
First norm:

\begin{align*}
    ||x_1||_1 &= \sum_{1}^{3} x_{1i} \\
              &= 2 - 3 - 1 \\
              &= \textbf{-2}
\end{align*}

\noindent and

\begin{align*}
    ||x_2||_1 &= \sum_{1}^{3} x_{2i} \\
              &= 1 + 1 - 1 \\
              &= \textbf{1}
\end{align*}


\noindent\textbf{b)} \newline
Second norm:

\begin{align*}
    ||x_1||_2 &= \sqrt{\sum_{1}^{3} x_{1i}^2} \\
              &= \sqrt{2^2 + (- 3)^2 + (- 1)^2} \\
              &= \bm{\sqrt{14}}
\end{align*}

\noindent and


\begin{align*}
    ||x_2||_2 &= \sqrt{\sum_{1}^{3} x_{2i}^2} \\
              &= \sqrt{1^2 + 1^2 + (- 1)^2} \\
              &= \bm{\sqrt{3}}
\end{align*}


\noindent\textbf{b)} \newline


% --------------------------------------------------------------
%     You don't have to mess with anything below this line.
% --------------------------------------------------------------
 
\end{document}
